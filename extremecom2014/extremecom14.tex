\documentclass{sig-alternate}
%\documentclass{sig-alternate-10pt}
 
\usepackage{verbatim}
\usepackage{graphics}
\usepackage{color}
\usepackage{url}
\usepackage{subfigure}
\usepackage{mdwlist}
\usepackage{floatflt}

\begin{document}

\conferenceinfo{ExtremeCom}{2014}
\CopyrightYear
\crdata

\title{The Breadcrumb Router: Bundle Trajectory Tracking and Geographic Source Routing in DTN
\thanks{\hrule\vspace{0.1in} This work was funded in part by the Laboratory
for Telecommunications Sciences, US Department of Defense, and
supported by the University of Maryland.  The opinions
expressed in this paper reflect those of the authors, and do not
necessarily represent those of the Department of Defense or US Federal
Government.}}

\author{
\alignauthor
Tomasz Kalbarczyk\textsuperscript{\#}\quad Brenton
Walker\textsuperscript{*}\quad Christine
Julien\textsuperscript{\#}\quad Angela
Hennessy\textsuperscript{*}\quad\\
Pedro
Santacruz\textsuperscript{\#}\quad Jonas Michel\textsuperscript{\#}\quad Amy
Alford\textsuperscript{\$}\quad \\  
\affaddr{\textsuperscript{\#}The University of Texas at Austin,
Austin, TX}\\
Email: \{tkalbar, pesantacruz, jonasrmichel, c.julien\}@utexas.edu\\
\affaddr{\textsuperscript{*}Laboratory for Telecommunications
Sciences, College Park, MD}\\
Email: brenton@ltsnet.net, ahennes1@math.umd.edu\\
\affaddr{\textsuperscript{\$}The University of Maryland, College Park,
MD}\\
Email: aloomis@math.umd.edu
}


%\numberofauthors{4} 
%\author{
%% 1st. author
%\alignauthor
%Tomasz Kalbarczyk, Pedro Santacruz, Jonas Michel, and Christine Julien\\
%       \affaddr{University of Texas-Austin}\\
%      \email{\{tkalbar, pesantacruz, jonasrmichel, c.julien\}@utexas.edu}
%% 2nd author
%\alignauthor
%Brenton Walker and Angela Hennessy\\
%       \affaddr{Laboratory for Telecommunications Sciences}\\
%       \affaddr{College Park, MD, USA}\\
%       \email{\{brenton, calvin\}@ltsnet.net}
%% 3rd author
%\alignauthor
%Amy Alford\\
%       \affaddr{University of Maryland}\\
%       \affaddr{College Park, MD, USA}\\
%       \email{aloomis@math.umd.edu}
%}

\maketitle

\begin{abstract}
\begin{sloppypar}
Since the dawn of time, man has wanted to know where the heck his bundles went, and to send them back on a specific geo-coded route.  Our new BreadCrumb Router and its associated extension block processors satisfiy this cosmic desire of human existence.  
\end{sloppypar}
\end{abstract}

\category{C.2.1}{Network Architecture and Design}{Store and forward networks}
%\terms{Design, Experimentation, Performance}
\keywords{Delay-Tolerant Networks, Bundle Protocol, Geographic routing}

% -----------------------------------------------
%
%   Introduction
%
% -----------------------------------------------
\section{Introduction}
Applications in delay-tolerant networks (DTNs) often desire to both track the path(s) of data through the network and to directly influence the movement of the data. DTNs are almost always integrated into some physical space that also influences that movement of nodes, data, and the phenomena about which the nodes communicate. In traditional IP networks, the {\bf traceroute} tool and {\bf source routing} protocols have assisted in {\em tracking} and {\em directing} packets of data, but the focus has traditionally been on movement through the logical network and not through physical space.

When the physical and logical intertwine, there is a desire to for data movement to reflect various aspects of the physical space the data inhabits. We address the dual challenges of {\em tracking} data as it moves through space and time and intentionally {\em routing} data through space and time. As an exemplar of tracking, consider the need for {\em data provenance} in sensing aggregation. As an aggregate collects information sensed about a physical phenomenon, the aggregate may need to dynamically compute the data {\em coverage} by tracking where the aggregate has traveled and collected information~\cite{michel12:spatiotemporal}. As an exemplar of the routing challenge, consider a piece of data that measures the concentration of a gas leak. Users in the area where the gas is expected to dissipate should be warned; this can be accomplished by associating the data item with a route through space and time that captures this expected dissipation. Tracking and routing can also be combined; imagine a generic scenario in which a {\em publisher} generates a piece of data that tracks its movement {\em en route} to a {\em subscriber}. Upon receiving the publication, the subscriber sends a response that must follow the reverse path of the original publication. This generic situation is a stand-in for a variety of concrete applications. For example, the original publication may have reserved some resources along the routing path that the return response relies on. We refer to this general situation as a maze traversal; a prisoner in the maze sends a probe that tracks its path as it attempts to exit the maze. When it reaches the exit, the responder (e.g. a rescuer) sends a response packet back to the prisoner along the same path through the maze.

Our problem of tracking differs substantially from the goals of existing utilities, not only in terms of the transition from logical network hops to physical spaces but also because the route a bundle in a DTN takes may not be stable (one bundle may pass through a given sequence of nodes, while a bundle sent just a minute later may take a different route). It therefore may often be important to track the route of {\em each} bundle. In our tracking facility, we track {\em both} logical network hops and the sequence of geographic locations it visits (whether because a device at one location transmits the bundle to a device at a different location or because the device holding the bundle moves). Besides being an illuminating diagnostic tool to understand the behavior of a DTN, tracking a bundle's geographic route can capture important meta-information related to the bundle's contents, as motivated above. 

Source routing, in which a packet carries with it the specific network hops it must traverse, and geographic routing, where packet routing is based on physical locations, have been popular in mobile ad hoc networks~\cite{johnson96:dynamic, karp00:gpsr}. We combine these approaches into a geographically informed version of source routing. Previous approaches to geographic routing predominantly use a {\em greedy} approach in which locally optimal decisions are used to move a bundle incrementally closer to its destination. Our approach to {\em geo-source routing} differs in that it allows the sender to pre-specify a sequence of geo-locations that serve as routing {\em waypoints}. This style of approach can solve a variety of challenges associated with traditional geographic routing. For example by explicitly directing the geographic path of a bundle, our protocol can explicitly route around known dead-ends in the network or around known areas of congestion in the network. Combining this style of geo-routing with other approaches could, for example, ensure that network coded bundles~\cite{petz11:network, widmer05:network} take sufficiently diverse routes through the network.

We introduce the {\sc breadcrumb} router, which implements geo-source routing on DTN bundles that can pre-specify their own delivery paths by providing a combination of geo-locations and logical network hops. 
%In our {\sc breadcrumb} router, elements in this sequence can be {\em required}, meaning that the bundle must touch the specified logical or physical location or {\em optional}, meaning that the sequence serves as a suggestion that the {\sc breadcrumb} router can use simply to guide its decision making. 
Each geo-location comes with a {\em margin of error}, which allows the {\sc breadcrumb} router to be satisfied by getting the bundle {\em near} the specified location without exactly reaching it. We introduce two data structures ({\em extension blocks}, in DTN parlance) that can be associated with bundles. The {\em GeoRouting} block holds the sequence of logical and geographical locations used to support geo-source routing. The {\em GeoTracking} block allows any bundle to collect a sequence of locations it visits in both logical and physical space. We describe these three components conceptually and show how we have implemented them in the IBR-DTN implementation of the bundle routing protocol~\cite{IBR-DTN-WASA}. We connect the {\sc breadcrumb} router to our existing Java-based implementation of {\em spatiotemporal trajectories}~\cite{michel12:spatiotemporal}, in which applications can perform expressive computations over data items given knowledge of their movements in space and time and geo-tracking and geo-routing of bundles on a pair of mobility scenarios.



%{\color{blue}
%Since the dawn of time, man has wanted to know where the heck his bundles went, and to send them back on a specific geo-coded route.  Our new BreadCrumb Router and its associated extension block processors satisfy this cosmic desire of human existence.  

%The {\bf traceroute} tool is a staple of traditional IP networks.  It allows a user or administrator to discover the sequence of IP routers their packets pass through between a specific source and destination.  Traceroute works by taking advantage of existing IGMP hop-count and reporting requirements.  Bundle Protocol as defined in RFC5050 has no inherent facility that can achieve this.  Furthermore, because the route a bundle takes may not be stable (one bundle may pass through one sequence of nodes, and a bundle sent a minute later may take another route) it is more salient in a DTN to comprehensively track the entire route of a single bundle, rather than send a series of bundles to probe the network.  Therefore bundle tracking functionality must be built fully as an extension to BP.  Additionally, since in many use cases the geographic mobility of the nodes in a DTN plays an integral part in the routing and delivery of a bundle, tracking the geographic route a bundle follows can be at least as interesting, and possibly {\it more} salient than the logical hops the bundle passes through.

%Besides being an illuminating diagnostic tool to understand the behavior of a DTN, tracking the geographic route of a bundle can capture important meta-information related to the contents of a bundle.  The simplest example of this is simply geo-tagging the location where a bundle originated.  A more involved example is tinkerpopping the datums of an oil slick with trajectories with a graph database as described in \cite{jonas-paper} and to be elaborated on here and in the following sections by Jonas and Christine.  Datums: it's the plural of ``datum'' when you're talking about geodetic information, dammit!

%Converse to simply geotracking a bundle, we build a prototype router for source geo-routing of a bundle.  In traditional IP networks, extensions to support source routing are defined for both IPv4 and IPv6, but are rarely used and often not supported or blocked.  On the other hand source routing is an integral part of some MANET routing protocols \cite{some-DSR-paper}.  Some reasons source routing could be used are to probe the structure of a network, or to specify that packets should avoid sections of a network that are known to be problematic because of congestion, reliability, or security concerns.  Implementing this type of {\it logical} hop-based source routing in DTN is certainly feasible, but in a dynamic mobile network is is difficult to imagine that a source will know exactly which intermediate nodes will be available to ferry its bundles.  Considering the importance of the geographic path in DTN routing, we propose geographic source routing.  In this system a sender can specify both logical (intermediate nodes) and geographic waypoints a bundle must pass through on its way to the destination.

%Geographic routing has been proposed and used in many DTN scenarios \cite{paper1,paper2,paper3,paper4,paper5,paper6}.  In most cases greedy geographic routing is used locally to move a bundle closer to a destination.  In other cases large-scale logical hops over an infrastructure network are used to get a bundle to a general geographic area of interest, and then other types of routing take precedence to get the bundle to its final destination.  Source geo-routing is different in that it allows a sender to pre-specify solutions to problems that can arise from greedy geographic routing.  Some of these issues and goals are:
%\begin{itemize}
%  \item avoiding known dead-ends in the network. I.e. local minima in the greedy geographic routing heuristic.
%  \item avoiding geographic areas where unreliability or congestion is anticipated.  For example if a shortest geographic route goes past a baseball stadium, but a game is scheduled for that day, a source may request that a bundle be routed around the expected traffic jam.
%  \item ensuring that network coded bundles take sufficiently diverse routes.  Since network coded information can be viewed as a shared secret problem, ensuring that not all of the encodings are ever in the same geographic area could be a security measure in a DTN.
%\end{itemize}
%\footnote{Brenton's original text, a lot of which gets reused in the above structure}
%}


\subsection {Prior Work}


% -----------------------------------------------
%
%   Use Cases
%
% -----------------------------------------------
% -----------------------------------------------
%
%   Use Cases
%
% -----------------------------------------------

{\bf Use Cases.} There are myriad uses for the {\em GeoTracking} and
{\em GeoRouting} bundle extensions, as well as for the {\sc
  breadcrumb} router itself. We briefly describe three examples that
separate the motivations for tracking, geo-routing, and the
combination of the two.

{\it Safe Data Sharing.} Tracking bundles is clearly widely useful for
debugging and general spatiotemporal provenance. As a concrete
example, a secure application may want to track the movement of
bundles through space to ensure that bundles (and their content) never
leave a pre-specified ``safe'' zone~\cite{michel12:spatiotemporal}. By
placing a {\em GeoTracking} extension on each bundle, the application
can compare the trajectory of the bundle against the safe zone to
ensure the desired property.

{\it An Oil Spill.} Consider an environmental disaster, e.g., an oil
spill, detected by one or more distributed sensors. Based on the
location of the spill and simultaneously sensed ambient information
(e.g., water currents, winds, etc.), the sensor can compute where the
spill is likely to dissipate and generate bundles that can be
explicitly routed along the trajectories of dissipation. As the
bundles propagate, their routing paths (i.e., trajectories) can be
updated by devices they pass through based on locally sensed ambient
conditions (e.g., changes in water currents or winds). Our {\em
  GeoRouting} extension to the Bundle protocol's block format,
combined with our {\sc breadcrumb} router, can achieve this behavior.

{\it A Maze.} To bring tracking and routing together, consider a
prisoner is held in a maze, and the maze is patrolled by guards who
can act as intermediate DTN nodes. The prisoner sends a bundle with a
destination of a known rescuer outside of the maze. This bundle tracks
its trajectory, logging a successful path out of the maze. The rescuer
can send a response back to the prisoner, routed along the waypoints
of the turns in the maze. The response may contain information about
the perils of the exit path, sensed as the prisoner's original message
traversed to the rescuer. Using our {\sc breadcrumb} router, the
rescuer's return bundle can reach the prisoner regardless of whether
the particular guards in the maze change, as long as there exists
enough coverage of the maze path by {\em some} DTN nodes.










% -----------------------------------------------
%
%   Architecture and Implementation
%
% -----------------------------------------------
\section{Architecture and Implementation}

Here we will describe the design, implementation, and architectural considerations of the GeoTracking and GeoRouting blocks, and the Breadcrumb Router.  All three of these componets involve tying together GPS information that is usually available in a platform-dependent way and is constantly changing, into the BPA.  This is espacially challenging to implement in the IBR-DTN architecture, since all bundle information is held in persistent storage and must be access through bundle filters, and the design ethos dictates that we not create any RAM-based data structures that would grow with the number of bundles held.

% -----------------------------------------------
%   GeoTracking Block
% -----------------------------------------------
% -----------------------------------------------
%   GeoTracking Block
% -----------------------------------------------
\subsection{The GeoTracking Block}
We enabled per-bundle tracking by defining the {\em GeoTracking} extension block, which, when attached to any bundle, collects both the logical hops that the bundle traverses and the bundle's trajectory through physical space. The {\em GeoTracking} block is a series of tracking entries (each one either recording a logical hop or a physical location) prefaced by a small header containing parameters for maintaining the block and counting the number of its entries.
\begin{figure}
\begin{center}
\includegraphics[width=\columnwidth]{figures/tracking-block.pdf}
\end{center}
\vspace{-.75cm}
\caption{Format of the GeoTracking Block}
\label{fig:tracking-block}
\vspace{-.5cm}
\end{figure}

\begin{sloppypar}
Figure~\ref{fig:tracking-block} depicts the {\em GeoTracking} block's format. The Block Header fields are specified by RFC5050\footnote{\scriptsize\url{http://tools.ietf.org/html/rfc5050}} and apply to any block in a bundle. In IBR-DTN, the block processor only deals with the contents of the block itself, and the BPA strips off the block header. Each {\em GeoTracking} block contains three mandatory fields:
\begin{description*}
  \item[Flags.] The flags tell intermediate BPAs what information to append to the block. The flags are: {\bf TRACK\_HOPS} (0x01), {\bf TRACK\_GEO} (0x02), and {\bf TRACK\_TIMESTAMP} (0x04).
  \item[Interval.] The interval (in seconds) tells intermediate BPAs the frequency with which to append a new GEODATA tracking entry to the entry list.  
  \item[Entry Count.] The entry count keeps track of the number of tracking entries contained in the block.
\end{description*}
\end{sloppypar}

Keeping each bundle's {\em GeoTracking} block updated as the device storing the bundle moves is non-trivial; keeping a {\em GeoTracking} block up-to-date in real time is not feasible for several reasons. First, there may be many bundles at a given BPA with {\em GeoTracking} blocks, and each bundle may have a different interval, requiring both responding to multiple timers and (in the case of IBR-DTN), reloading and storing each bundle from disk on every update. Instead, we maintain a global GPS log and updates a bundle's associated {\em GeoTracking} block only when a bundle is serialized for sending. Upon serialization, we examine the history of the GPS log and attach all of the appropriate entries to the {\em GeoTracking} block. Maintaining the GPS log is also non-trivial. To completely satisfy any arbitrary tracking interval requirement would require recording the node's location at an interval of the GCD of all of the tracking intervals, which may not be know {\em a priori}. In our prototype, we choose a fixed global interval, and bundles can request a {\em less frequent} update. Finally is the question of where to maintain the GPS log. A BPA should be as platform-independent as possible, while GPS information acquisition is quite platform-dependent. Therefore, we assume a host-specific agent that logs GPS data to a file at the aforementioned interval. Each time a {\em GeoTracking} block is serialized, the BPA scans the log file for the necessary entries and creates the necessary tracking entries for the {\em GeoTracking} block.

This approach still has some drawbacks, especially in IBR-DTN.  First, it requires opening and reading a (potentially long) log file each time a {\em GeoTracking} block is serialized.  Second, in IBR-DTN, because there is no function to "finalize" the contents of a block prior to serializing, the GPS log must be parsed twice: once when the block processor calculates the block's length, and again when the actual serialization takes place.  
%In IBR-DTN both the {\bf getLength()} and {\bf serialize()} functions are {\bf const}, so it is not possible to modify any fields of the GeoTracking block itself to cache the state of the block when {\bf getLength()} is called.  
This technically creates a race condition between these two calls, where the GPS log may get longer between the two functions.  Resolving these issues completely may require some modifications to the serialization process of IBR-DTN and is reserved for future work.

%\subsubsection{GPS Coordinate Representation as SDNV} \label{gps-representation}
Our extension blocks represent GPS coordinates in signed degrees format, where latitude ranges from $-90^{\circ}$ to $90^{\circ}$ and longitude ranges from $-180^{\circ}$ to $180^{\circ}$.  Both latitude and longitude are considered to be type {\bf float}.  However, since the self-delimiting numeric values (SDNVs)\footnote{\scriptsize\url{http://tools.ietf.org/html/draft-irtf-dtnrg-sdnv-09}} used in the bundle protocol cannot represent floating point numbers or negative values, we make two transformations to encode longitude and latitude in the{\em GeoTracking} block.  If $\theta<0$ we compute $\theta^{\prime}=\theta+360^{\circ}$.  Then we scale all coordinates up by a factor of $1048576$.  This gives us at least 20 bits of precision in both values, which is more than enough for meter-level resolution in the GPS coordinates.  When the blocks are received and deserialized, these transforms are reversed to give the original floating point values.






% -----------------------------------------------
%   GeoRouting Block
% -----------------------------------------------
% -----------------------------------------------
%   GeoRouting Block
% -----------------------------------------------
\subsection{GeoRouting Block}

The GeoRouting block is an extension block designed to support source-routing based on either intermediate geographic waypoints, or logical hops (EIDs), or both.  We implement a specific router which uses it, but a wealth of other routing possibilities ideas spring to mind as well.  The GeoRouting block is essentially a list of GeoRoutingEntry blocks, each one containing an intermediate routing goal.

The main part of the block only contains two fields, the flags (which are currently unused) and an entry count specifying how many GeoRoutingEntry sub-blocks are to follow.  Each GeoRoutingEntry has several fields.  The first is a set of flags specifying the requirments and contents of the entry.  The four flags currently defined are:
\begin{description}
  \item[REQUIRED] If set, then this GeoRoutingEntry {\it must} be satisfied for the bundle to be considered delivered.  Otherwise the entry is consideered optional.
  \item[ORDERED] If set, this GeoRoutingEntry {\it must} be satisfied before any successive entries can be considered.  Otherwise an intermediate node can pop successive entries off the list before this one is satisfied.
  \item[GEO\_PRESENT] If set, this entry contains a latitude/longitude point to be used as an intermediate routing goal.  To satisfy this GeoRoutingEntry the bundle must at some point reside on a node that is within a factor of {\bf margin} of this GPS coordinate.
  \item[EID\_PRESENT] If set, this entry contains an EID.  To satisfy this GeoRoutingEntry the bundle must at some point reside on a node whose singleton EID matches the required EID.
\end{description}

If both {\bf GEO\_PRESENT} and {\bf EID\_PRESENT} are set, then the bundle must be carried by the node specified in the EID field to the location specified by the GPS coordinate.
An example of a case that would have both {\bf REQUIRED} and {\bf ORDERED} set to false is if the sender intends to allow the bundle to take some shortcuts if they are available.  In that case some entries could be discarded if the bundle finds itself close enough to a successive waypoint, or with an opportunity to get to a successive waypoint.  The specifics of how a router prioritizes the un-ordered and optional GeoRouting entries are up to the particular router.  Our router's approach will be described in the next section.

The {\bf margin} field specifies {\it how close} the block must come to the specified GPS coordinate for the entry to be considered satisfied.  The margin is a floating point value given in absolute degrees.  For example, if $m$ is the margin contained in the entry, and $x_0$ and $y_0$ are the required longitude and latitude, then getting the bundle to any point in the range $(x_0\pm m, y_0\pm m)$ will satisfy the entry.  We realize that this results in a target area that is roughly rectangular instead of circular, and that the margin of error will result in different {\it actual} margins of error at different points on he globe.  We implemented it this way to keep the router implementation simpler.  Interpreting the margin of error as a radius in meters would require more compliceted geodetic calculations in the router for each geo-routed bundle, each time a node's location changes.

In order to represent it as an SDNV we applu the same transform as with the GPS coorsinates, scaling by a factor of 1048576 and rounding to an integer.  We apply the opposite scaling when the block is processed to recover the floading point value.

\begin{figure}
\begin{center}
\includegraphics[width=.9\columnwidth]{figures/georouting-block.pdf}
\end{center}
\caption{Format of the GeoRouting Block}
\label{fig:georouting-block}
\end{figure}

The GeoRouting block is easier to maintain from the perspective of the block processor, but much more complicated for a router, at least in our implementation.  The router updates the extension block in the node's data store, and the block processor only needs to serialize the block as it appears, with no modifications.  The details of this type of update will be in section~\ref{routing}.





% -----------------------------------------------
%   The Breadcrumb Router
% -----------------------------------------------
% -----------------------------------------------
%   The Breadcrumb Router
% -----------------------------------------------
\subsection{The Breadcrumb Router}

The {\sc Breadcrumb} Router is designed to perform geo-source routing by making use of the {\sc GeoRouting} extension block attached to a bundle, the GPS coordinates of the host, and the GPS coordinates of neighboring peers. As described in the {\sc GeoRouting} section, the {\sc GeoRouting} block contains a list of entries which specify routing goals (specific nodes or GPS locations). For each bundle, the router takes the entry at the end of the list, and compares it against GPS coordinates of neighboring peers to decide if a peer should receive the bundle. The router also periodically compares this entry to the GPS coordinates of the host to decide if the entry should be popped from the list (since the bundle has now visited that location). Using this strategy the bundle eventually visits every location in the GeoRouting block. Since following these entries back to a particular location is analogous to following breadcrumbs, we have coined our router, the Breadcrumb Router.

In IBR-DTN, routers loop through a task queue in order to perform their tasks. As such, the {\sc Breadcrumb} Router's functionality can be separated into three tasks:

\begin{description}
\item[SearchNextBundle] Whenever the router receives an event indicating that data has changed regarding it's connection to a peer (for example, a successful handshake with a newly discovered peer or a completed bundle transfer), the router computes the set of bundles to send to the peer.
\item[NextExchange] The router needs GPS location information from all neighboring peers in order to guide routing decisions. This task, which is queued periodically, triggers this exchange.
\item[UpdateMyLocation] The router needs to maintain the location of the host in order to know when a {\sc GeoRouting} entry has been visited. This task is queued both periodically, and whenever a bundle is received to determine which entries to remove for each bundle.
\end{description}

\subsubsection{Bundle Filters and Meta Bundles}

The SearchNextBundle and UpdateMyLocation tasks both require inspecting the location information packaged in the GeoRouting block of each bundle. Performing the inspection efficiently using IBR-DTN presents a dilemma. Pulling each bundle, inspecting it, and potentially modifying it is inherently inefficient since the bundles are in persistent storage. On the otherhand, maintaining a data-structure in memory that contains all the Geo information pertaining to each bundle goes against the design of IBR-DTN. We devised an efficient solution that does not require the creation of an additional data-structure and minimizes the amount of bundles that need to pulled from persistent storage.

This solution uses two IBR-DTN constructs: {\sc bundle filters} and {\sc meta bundles}. {\sc Meta bundles} are light-weight representations of bundles that contain fields of particular interest. A meta bundle is generated for each bundle. {\sc Bundle filters} query the storage for a list of meta bundles that meet a set of rules. For our implementation we added three fields to each {\sc meta bundle}:

\begin{description}
\item[hasgeoroute] A boolean identifier indicating that the bundle has a {\sc GeoRouting} block
\item[nextgeohop] The last entry in the {\sc GeoRouting} block
\item[reacheddest] A boolean flag indicating that there are no more entries in the {\sc GeoRouting} block (i.e., the bundle has reached its ``final'' destination)
\end{description}

We also created two {\sc bundle filters}, one pertaining to each task that needs to inspect the bundles.

\begin{description}
\item[SearchNext] This filter is used to determine what bundles to send to each peer, so it is called in the {\sc SearchNextBundle} task. For each {\sc meta bundle} that {\sc hasgeoroute} is {\sc true}, the filter compares the location of the peer with the {\sc nextgeohop} to see if the peer is closer to it than the host. If so, the meta bundle is added to the list.
\item[UpdateLocation] This filter is used to determine which bundles need to have their {\sc GeoRouting} block entries updated, since the location in the last entry has been visited. It uses similar logic to the {\sc SearchNext} filter, but instead compares the location of the host to the {\sc nextgeohop} from the meta bundle. If the location of the host falls within a prespecified margin of the {\sc nextgeohop} location, the meta bundle is added to the list.
\end{description}

The key benefit of using the {\sc bundle filters} is that only one lookup into persistent storage is necessary to retrieve a list of meta bundles that need action. In the case of the {\sc SearchNextBundle} task, the list return by the corresponding filter is simply used to transfer bundles. In the {\sc UpdateMyLocation} task, each meta bundle in the list represents a bundle that needs to be pulled from storage to have its {\sc GeoRouting} block modified. However, since we added that {\sc nextgeohop} entry to each meta bundle for filtering purposes, we are guaranteed that each bundle that is pulled from storage actually needs the last entry in its {\sc GeoRouting} block to be popped off. For example, imagine that there are 10 bundles in storage, and the current location of the host satisfies the {\sc GeoRouting} block entry for one of these bundles. Our solution requires only two lookups into storage: one to retrieve the list of meta bundles that need modification (one in this case) and one to replace that bundle in storage. This is a considerable improvement over having to retrieve each bundle from storage just to inspect a {\sc GeoRouting} block that in the vast majority of cases will not need any updates.

\subsubsection{Additional Features}
Although the primary functionality of the router is geo-source routing, it also doubles as an epidemic router for all bundles that do not contain {\sc GeoRouting} blocks. This functionality was useful so that we could use a single router to both build our trajectories (lay the breadcrumbs) and to follow them back to the source (follow the breadcrumbs). The router also supports single-copy routing, so that for tracking purposes during testing, only a single bundle with a {\sc GeoRouting} block persists in the network.

\subsubsection{Future Improvements}
For our initial prototype, we focused on ordered, geo-source routing. Therefore, our router assumes that all {\sc GeoRouting} blocks contain entries that must be visited in order, and that each entry pertains to a particular GPS location. However, our {\sc GeoTracking} and {\sc GeoRouting} blocks also support unordered and hop entries. Supporting a hetereogenous series of entries would make the router more generalizable since it could be applied to scenarios where the order that locations are visited does not matter, or where a few specific nodes must be visited.

Although not specifically associated with our router, current IBR-DTN does not support updates to bundles in storage. Therefore, each update to a {\sc GeoRouting} block requires a separate remove and store operation. This is inefficient because it requires that the entire bundle be re-written to storage during every update (this actually makes our modifications to the meta bundles even more critical). Adding this update functionality would improve the performance of our router and of IBR-DTN storage in general.




% -----------------------------------------------
%
%   Experiments
%
% -----------------------------------------------
\section{Experiments}\label{sec:experiments}
To demonstrate our {\em GeoTracking} and {\em GeoRouting} extension
blocks and our {\sc breadcrumb} router, we used our implementations of
them in the IBR-DTN core and with the Java API and performed
end-to-end experiments on the VirtualMeshTest (VMT) channel-emulated
testbed~\cite{hahn10:using, kim11:reality}.  VMT uses Linux-based real
wireless nodes with commodity wireless hardware to emulate mobile
environments.  The wireless testbed is effectively an analog channel
emulator based on an array of programmable attenuators.  Given a
desired physical arrangement of nodes, the system computes the
expected path loss between nodes and programs the attenuators
accordingly.  VMT updates the attenuations every second to emulate a
mobile wireless environment.

For each of our experiments, we define one node to be a {\em
  sender} and another to be a {\em responder}. Using the Java IBR-DTN
bridge API, we used our application to create application-level
bundles that we sent through our router. The {\em sender} initially
creates a probe bundle, to which it attaches a {\em GeoTracking}
extension block. The {\em sender} then sends this bundle to the {\em
  responder} using his EID as the target of routing. When the {\em
  responder} receives this bundle at the application level, it creates
a bundle with a {\em GeoRouting} tracking block that specifies as
waypoints a subset of the locations visited by probe. The geo-routed
bundle is sent back along the waypoints.

%{\bf The Ladder.} The first scenario contains thirteen nodes. Eight of
%the nodes form two evenly spaces parallel lines; these form the sides
%of the ladder. One node sits, stationary, at the bottom of the ladder;
%another node sits, stationary, at the top of the ladder. Three nodes
%repeatedly climb from the bottom of the ladder to the top and back
%down. {\color{red} The sender is at the top of the ladder and the
 % responder at the bottom; Figure~\ref{fig:ladder1} shows the
  %trajectory of the responder's geo-routed bundle, with the target
  %location waypoints from the {\em GeoRouting} extension block
  %highlighted.}
%\begin{figure}
%\begin{center}
%\begin{tikzpicture}
%\begin{axis}[legend pos=south east]
%\addplot+[only marks,color=black,mark=*,mark size=0.2pt] 
%	table[x=x,y=y] {../data/ladder/pgy.txt};
%\addplot[only marks,color=red,mark=x] 
%	table[x=x,y=y] {../data/ladder/pgy_2.txt};
%\legend{Bundle Trajectory}
%\end{axis}
%\end{tikzpicture}
%\end{center}
%\vspace{-.75cm}
%\caption{Ladder Experiment 1}\label{fig:ladder1}
%\vspace{-.5cm}
%\end{figure}

{\bf Crop Circles.} In our first set of experiments, we created grids
of nodes that move continuously in circles in either a clockwise or
counter clockwise pattern. These {\em crop circles} consist of six
nodes arranged as in Figure~\ref{fig:cropcircle1}; the sender is the
node in the lower left; the responder is the node in the upper right.
Figure~\ref{fig:cropCirclesExperiment} shows the results of a single
execution on the crop circles network;
Figure~\ref{fig:cropCirclesExperiment}(a) shows the trajectory of the
probe bundle sent from the sender to the responder, while
Figure~\ref{fig:cropCirclesExperiment}(b) shows the trajectory
followed by the return (geo-routed) bundle. The small circles indicate
the waypoints specified in the {\em GeoRouting} block of the
responder's bundle (which were computed automatically by our
application layer from the probe's {\em GeoTracking} block).
\begin{figure}[!h]
\vspace{-.2cm}
\begin{center}
%\includegraphics[width=\columnwidth+.5]{figures/cropcircle1.png}
\includegraphics[width=.65\columnwidth]{figures/cropcircle1.png}
\end{center}
\vspace{-.75cm}
\caption{Crop circles mobility scenario}\label{fig:cropcircle1}
\vspace{-.25cm}
\end{figure}

The track of the sender's probe bundle stops as soon as the responder
(the node in the upper right corner) receives the bundle. When the
responder generates its {\em GeoRouting} block for the return bundle,
it inserts its location as the first waypoint. The track of the
responder's bundle (Figure~\ref{fig:cropCirclesExperiment}(b)) hits
all of the waypoints specified in the {\em GeoRouting} bundle. In the
figure, the final waypoint does not appear to quite be reached. This
is because our experiments deserialize and log the tracking blocks
only when a bundle arrives at a node; once the node in the top left
corner hands the bundle to the node in the lower left corner, the
latter node continues to move along the path shown in
Figure~\ref{fig:cropcircle1}, eventually crossing that final waypoint.
\begin{figure}[!h]
\begin{center}
\includegraphics[width=.8\columnwidth]{figures/CropCirclesExperiment2.png}\\
(a)\\
\includegraphics[width=.8\columnwidth]{figures/CropCirclesExperiment.png}\\
(b)\\
\end{center}
\vspace{-.5cm}
\caption{Crop circles tracking and trajectories. (a) The tracked
  trajectory of the probe bundle. (b) The tracked trajectory of the
  geo-routed bundle.}
\label{fig:cropCirclesExperiment}
\vspace{-.25cm}
\end{figure}


{\bf The Maze.} Our final experiments mirror the motivating scenario
of the prisoner in the maze. The maze consists of a series of
connected hallways, each patrolled by a single guard who moves back
and forth along hallway. Our maze is shown in
Figure~\ref{fig:maze}. The sender (i.e., the prisoner) is the node in
the lower left corner, while the responder (i.e., the rescuer) is the
node in the lower right corner.
\begin{figure}
\begin{center}
\includegraphics[width=.7\columnwidth]{figures/newMaze.pdf}
\end{center}
\vspace{-.75cm}
\caption{Maze mobility scenario}
\label{fig:maze}
\vspace{-.25cm}
\end{figure}

Figure~\ref{fig:mazeExperiment} shows the results of a single
execution on the maze in Figure~\ref{fig:maze}. The figure shows only
the trajectory of the rescuer's bundle that is geo-routed back to the
prisoner based on the tracked trajectory of the prisoner's probe
bundle. While copies of the prisoner's probe bundle do wander down the
maze's dead-end paths, the reply from the rescuer reflects only the
successful path through the maze. In Figure~\ref{fig:mazeExperiment},
The responder's bundle successfully reaches all of the {\em
  GeoRouting} block's required waypoints without following any
detours; the last waypoint is reached eventually, when the prisoner
moves back along his corridor.
\begin{figure}
\begin{center}
\includegraphics[width=.8\columnwidth]{figures/MazeExperiment.png}
\end{center}
\vspace{-.75cm}
\caption{Maze routing trajectory}
\label{fig:mazeExperiment}
\vspace{-.5cm}
\end{figure}




% -----------------------------------------------
%
%   Conclusion
%
% -----------------------------------------------
\section{Conclusion}




% Once the sections and outline are solidified, breaking the paper up into separate files will be useful.  For now I'll put the sections inline.
%
%% -----------------------------------------------
%
%   Introduction
%
% -----------------------------------------------
\section{Introduction}
Applications in delay-tolerant networks (DTNs) often desire to both track the path(s) of data through the network and to directly influence the movement of the data. DTNs are almost always integrated into some physical space that also influences that movement of nodes, data, and the phenomena about which the nodes communicate. In traditional IP networks, the {\bf traceroute} tool and {\bf source routing} protocols have assisted in {\em tracking} and {\em directing} packets of data, but the focus has traditionally been on movement through the logical network and not through physical space.

When the physical and logical intertwine, there is a desire to for data movement to reflect various aspects of the physical space the data inhabits. We address the dual challenges of {\em tracking} data as it moves through space and time and intentionally {\em routing} data through space and time. As an exemplar of tracking, consider the need for {\em data provenance} in sensing aggregation. As an aggregate collects information sensed about a physical phenomenon, the aggregate may need to dynamically compute the data {\em coverage} by tracking where the aggregate has traveled and collected information~\cite{michel12:spatiotemporal}. As an exemplar of the routing challenge, consider a piece of data that measures the concentration of a gas leak. Users in the area where the gas is expected to dissipate should be warned; this can be accomplished by associating the data item with a route through space and time that captures this expected dissipation. Tracking and routing can also be combined; imagine a generic scenario in which a {\em publisher} generates a piece of data that tracks its movement {\em en route} to a {\em subscriber}. Upon receiving the publication, the subscriber sends a response that must follow the reverse path of the original publication. This generic situation is a stand-in for a variety of concrete applications. For example, the original publication may have reserved some resources along the routing path that the return response relies on. We refer to this general situation as a maze traversal; a prisoner in the maze sends a probe that tracks its path as it attempts to exit the maze. When it reaches the exit, the responder (e.g. a rescuer) sends a response packet back to the prisoner along the same path through the maze.

Our problem of tracking differs substantially from the goals of existing utilities, not only in terms of the transition from logical network hops to physical spaces but also because the route a bundle in a DTN takes may not be stable (one bundle may pass through a given sequence of nodes, while a bundle sent just a minute later may take a different route). It therefore may often be important to track the route of {\em each} bundle. In our tracking facility, we track {\em both} logical network hops and the sequence of geographic locations it visits (whether because a device at one location transmits the bundle to a device at a different location or because the device holding the bundle moves). Besides being an illuminating diagnostic tool to understand the behavior of a DTN, tracking a bundle's geographic route can capture important meta-information related to the bundle's contents, as motivated above. 

Source routing, in which a packet carries with it the specific network hops it must traverse, and geographic routing, where packet routing is based on physical locations, have been popular in mobile ad hoc networks~\cite{johnson96:dynamic, karp00:gpsr}. We combine these approaches into a geographically informed version of source routing. Previous approaches to geographic routing predominantly use a {\em greedy} approach in which locally optimal decisions are used to move a bundle incrementally closer to its destination. Our approach to {\em geo-source routing} differs in that it allows the sender to pre-specify a sequence of geo-locations that serve as routing {\em waypoints}. This style of approach can solve a variety of challenges associated with traditional geographic routing. For example by explicitly directing the geographic path of a bundle, our protocol can explicitly route around known dead-ends in the network or around known areas of congestion in the network. Combining this style of geo-routing with other approaches could, for example, ensure that network coded bundles~\cite{petz11:network, widmer05:network} take sufficiently diverse routes through the network.

We introduce the {\sc breadcrumb} router, which implements geo-source routing on DTN bundles that can pre-specify their own delivery paths by providing a combination of geo-locations and logical network hops. 
%In our {\sc breadcrumb} router, elements in this sequence can be {\em required}, meaning that the bundle must touch the specified logical or physical location or {\em optional}, meaning that the sequence serves as a suggestion that the {\sc breadcrumb} router can use simply to guide its decision making. 
Each geo-location comes with a {\em margin of error}, which allows the {\sc breadcrumb} router to be satisfied by getting the bundle {\em near} the specified location without exactly reaching it. We introduce two data structures ({\em extension blocks}, in DTN parlance) that can be associated with bundles. The {\em GeoRouting} block holds the sequence of logical and geographical locations used to support geo-source routing. The {\em GeoTracking} block allows any bundle to collect a sequence of locations it visits in both logical and physical space. We describe these three components conceptually and show how we have implemented them in the IBR-DTN implementation of the bundle routing protocol~\cite{IBR-DTN-WASA}. We connect the {\sc breadcrumb} router to our existing Java-based implementation of {\em spatiotemporal trajectories}~\cite{michel12:spatiotemporal}, in which applications can perform expressive computations over data items given knowledge of their movements in space and time and geo-tracking and geo-routing of bundles on a pair of mobility scenarios.



%{\color{blue}
%Since the dawn of time, man has wanted to know where the heck his bundles went, and to send them back on a specific geo-coded route.  Our new BreadCrumb Router and its associated extension block processors satisfy this cosmic desire of human existence.  

%The {\bf traceroute} tool is a staple of traditional IP networks.  It allows a user or administrator to discover the sequence of IP routers their packets pass through between a specific source and destination.  Traceroute works by taking advantage of existing IGMP hop-count and reporting requirements.  Bundle Protocol as defined in RFC5050 has no inherent facility that can achieve this.  Furthermore, because the route a bundle takes may not be stable (one bundle may pass through one sequence of nodes, and a bundle sent a minute later may take another route) it is more salient in a DTN to comprehensively track the entire route of a single bundle, rather than send a series of bundles to probe the network.  Therefore bundle tracking functionality must be built fully as an extension to BP.  Additionally, since in many use cases the geographic mobility of the nodes in a DTN plays an integral part in the routing and delivery of a bundle, tracking the geographic route a bundle follows can be at least as interesting, and possibly {\it more} salient than the logical hops the bundle passes through.

%Besides being an illuminating diagnostic tool to understand the behavior of a DTN, tracking the geographic route of a bundle can capture important meta-information related to the contents of a bundle.  The simplest example of this is simply geo-tagging the location where a bundle originated.  A more involved example is tinkerpopping the datums of an oil slick with trajectories with a graph database as described in \cite{jonas-paper} and to be elaborated on here and in the following sections by Jonas and Christine.  Datums: it's the plural of ``datum'' when you're talking about geodetic information, dammit!

%Converse to simply geotracking a bundle, we build a prototype router for source geo-routing of a bundle.  In traditional IP networks, extensions to support source routing are defined for both IPv4 and IPv6, but are rarely used and often not supported or blocked.  On the other hand source routing is an integral part of some MANET routing protocols \cite{some-DSR-paper}.  Some reasons source routing could be used are to probe the structure of a network, or to specify that packets should avoid sections of a network that are known to be problematic because of congestion, reliability, or security concerns.  Implementing this type of {\it logical} hop-based source routing in DTN is certainly feasible, but in a dynamic mobile network is is difficult to imagine that a source will know exactly which intermediate nodes will be available to ferry its bundles.  Considering the importance of the geographic path in DTN routing, we propose geographic source routing.  In this system a sender can specify both logical (intermediate nodes) and geographic waypoints a bundle must pass through on its way to the destination.

%Geographic routing has been proposed and used in many DTN scenarios \cite{paper1,paper2,paper3,paper4,paper5,paper6}.  In most cases greedy geographic routing is used locally to move a bundle closer to a destination.  In other cases large-scale logical hops over an infrastructure network are used to get a bundle to a general geographic area of interest, and then other types of routing take precedence to get the bundle to its final destination.  Source geo-routing is different in that it allows a sender to pre-specify solutions to problems that can arise from greedy geographic routing.  Some of these issues and goals are:
%\begin{itemize}
%  \item avoiding known dead-ends in the network. I.e. local minima in the greedy geographic routing heuristic.
%  \item avoiding geographic areas where unreliability or congestion is anticipated.  For example if a shortest geographic route goes past a baseball stadium, but a game is scheduled for that day, a source may request that a bundle be routed around the expected traffic jam.
%  \item ensuring that network coded bundles take sufficiently diverse routes.  Since network coded information can be viewed as a shared secret problem, ensuring that not all of the encodings are ever in the same geographic area could be a security measure in a DTN.
%\end{itemize}
%\footnote{Brenton's original text, a lot of which gets reused in the above structure}
%}
%\input{background}
%\input{routing}
%\input{evaluationPharos}
%\input{evaluationNYC}
%\input{evaluationVMT}
%\input{evaluationBrief}
%\input{conclusion}
%\input{related}
% removed in favor of combining related work with "Background" section
%\section*{Acknowledgments}
%This work was funded in part by the US Dept. of Defense. The views expressed are those of the authors and may not necessarily reflect the views of the sponsoring agencies.\\

\bibliographystyle{abbrv}
%\begin{scriptsize}
\bibliography{geobreadcrumb}
%\end{scriptsize}

\end{document}

