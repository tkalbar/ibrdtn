\begin{abstract}
\begin{sloppypar}
  In delay-tolerant networks, generally, geographic knowledge and influence often emerge as key components of moving data around the network, but existing DTN artifacts focus almost exclusively on the networking aspects of moving bundles and not on the more inherently physical concepts of space and time. In this paper, we look at the protocols and data structures necessary to add flexible and expressive support for geographic tracking and routing to delay-tolerant networks. On the one hand, our {\em GeoTracking} extension block allows for the implementation of expressive tracking of bundles' movements through space and time. On the other hand, our {\em GeoRouting} extension block and an associated {\sc breadcrumb} router show how expressive space-time information can also be used to direct bundles through the delay-tolerant network. We start by motivating the conceptual underpinnings of GeoTracking and GeoRouting, then we describe how we integrate this functionality into the IBRDTN implementation of the bundle routing protocol. We empirically demonstrate our approaches on a set of network scenarios emulated in a DTN network emulator.

%Since the dawn of time, man has wanted to know where the heck his bundles went, and to send them back on a specific geo-coded route.  Our new BreadCrumb Router and its associated extension block processors satisfiy this cosmic desire of human existence.  
\end{sloppypar}
\end{abstract}