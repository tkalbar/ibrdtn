\section{Motivation}
{\bf Related Work.}
Our {\sc breadcrumb} router combines geographic routing and source routing to enable nodes in a DTN to explicitly specify {\em waypoints} that a bundle should ``hit'' as it moves from a source to a destination. These waypoints (as well as the final destination) may be a combination of logical addresses (e.g., node identifiers) and physical locations. In traditional source routing, e.g.,~\cite{johnson96:dynamic}, each packet carries the sequence of node identifiers (e.g., IP addresses) that completely specifies the exact path the packet should follow. On the other hand, in common geographic-based routing protocols, e.g.,~\cite{florian13:overdrive, karp00:gpsr, navas97:geocast} packets are routed, usually greedily, with the destination being a physical coordiante that specifies a geographic location. These greedy algorithms have been increasingly fortified to mitigate the impact of the non-optimality of the greedy decisions, for example to route around topology holes~\cite{tian03:spatially}. Some efforts have been made to combine source routing and geographic routing but with a focus on how one can assist the other, for example by using location to reduce the overhead of discovering source routes~\cite{basagni99:dynamic}. Our combination, on the other hand, is motivated by examples in which an application desires that the routing task forces the packet to follow a pre-specified route through physical space.  In vehicular ad hoc networks (VANETs), routing algorithms combat the challenges of urban scenarios, e.g., in areas dense with buildings, packets must often be routed around buildings and other obstructions of radio signals. These VANET routing protocols use {\em junctions} as waypoints to help packets navigate around obstacles~\cite{jerbi07:improved, lochert05:geographic}, but these approaches rely on {\em a priori} city street maps or knowledge about vehicular traffic patterns to bootstrap effective communication.

Early work on trajectory-based forwarding focuses on generating and following alternative routes around congested areas~\cite{niculescu03:TBF}. Our work involves following a mixture of addresses and waypoints rather than generating an approximate path. A more recent Predict and Relay approach~\cite{yuan09:PRE} leverages the premise that nodes often revisit waypoints in order to improve end-to-end delivery in DTNs. This work is orthogonal to ours since it improves the likelihood that bundles will be forwarded to nodes that visit waypoints along our trajectory.

Our routing technique also shares features with the rumor routing algorithm~\cite{braginsky02:RRA}. The algorithm generates several paths to events (trajectories). Arbitrary nodes send bundles down random paths until they reach a node along the trajectory, at which point the bundle can be forwarded to the event. A key difference in our approach is that our bundles are always making progress towards the next waypoint, and we don't presume that the path between any two waypoints is previously known or static (i.e., there is no event path).

Work targeting DTNs has incorporated geographic information into DTN bundle routing. GeoSpray~\cite{soares14:geospray} augments traditional greedy geographic forwarding with a store-and-forward behavior that improves delivery success in intermittently connected networks. Other approaches use predictions based on navigation systems~\cite{cheng10:geodtn} or explicitly relax requirements associated with location knowledge to enable geographic based routing when only partial location information is available~\cite{kuiper11:geographical}. Finally, some DTN approaches explicitly rely on the support of known infrastructure (e.g., kiosks at bus stops) to reliably route based on position information~\cite{park12:position}. In contrast to these approaches, we rely on completely distributed and ad hoc behavior. Further, as described next, our use cases are somewhat divergent in that we assume the need for the content carried by a bundle to reach specific waypoints, as opposed to only using the waypoints to mitigate routing challenges.
