\section{Motivation}
\subsection{Related Work}
The {\sc breadcrumb} router we describe in this paper is built on the observation that the combination of geographic routing and source routing can enable nodes in a DTN to explcitly specify {\em waypoints} that a bundle should ``hit'' as it moves from a source to a destination. Further, these waypoints (as well as the final destinations) may be a combination of logical addresses (e.g., node identifiers) and physical locations. In traditional source routing, e.g.,~\cite{johnson96:dynamic}, each routed packet carries with it the sequence of node identifiers (e.g., IP addresses) that completely specifies the exact path the packet should follow. On the other hand, in common geographic-based routing protocols, e.g.,~\cite{florian13:overdrive, karp00:gpsr, navas97:geocast} packets are routed, usually greedily, with the destination being not a network address but a set of coordinates that specify a geographic location. These greedy algorithms have been increasingly fortified by efforts to mitigate the impact of the non-optimality of the greedy decisions, for example to route around topology holes~\cite{tian03:spatially}. Some efforts have been made to combine source routing and geographic routing, but usually focus on how one can assist the other, for example by using location coordinates to reduce the overhead of discovering source routes~\cite{basagni99:dynamic}. Our combination, on the other hand, is motivated by examples in which an application desires that the routing task attempt to force the packet to follow a pre-specified route through physical space.

In vehicular ad hoc networks (VANETs), routing algorithms have emerged to combat the challenges in urban scenarios, specifically that in areas dense with buildings, packets must often be routed around buildings and other obstructions of radio signals. These VANET routing protocols use {\em junctions} as waypoints to help packets navigate around obstacles~\cite{jerbi07:improved, lochert05:geographic}; these approaches often rely on {\em a priori} city street maps or knowledge about comment vehicular traffic patterns to bootstrap effective communication.

More recently, work specifically targetting DTNs has looked at augmenting relatively standard approaches to DTN bundle routing with geographic information. For instance, GeoSpray~\cite{soares14:geospray} augments a traditional greedy geographic forwarding algorithm with a store-and-forward behavior that improves delivery success in intermittently connected networks. Other approaches use predictions based on navigation systems to direct bundles~\cite{cheng10:geodtn} or explicitly relax requirements associated with location knowledge to enable geographic based routing when only partial location information is available~\cite{kuiper11:geographical}. Finally, like some VANET approaches, some DTN approaches explicitly rely on the support of known infrastucture (e.g., kiosks at bus stops) to reliably route based on position information~\cite{park12:position}. In contrast to these approaches, we rely on completely distributed and ad hoc behavior. Further, as described next, our use cases are somewhat divergent in that we assume the need for the content carried by a bundle to reach specific waypoints, as opposed to only using the waypoints to mitigate routing challenges.