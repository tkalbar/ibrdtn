\section{Related Work}

GPSR~\cite{karp00:gpsr} and DSR~\cite{johnson96:dynamic}

location coordinates have been used in source routing to reduce the overhead of generating source routes~\cite{basagni99:dynamic}

geocasting specifically addresses packets to nodes at a location (instead of by logical address)~\cite{florian13:overdrive, navas97:geocast}

augmentations that include spatially-informed forwarding to attempt to mitigate the impact of, e.g., topology holes~\cite{tian03:spatially}

junction-aware routing in vehicular ad hoc networks~\cite{jerbi07:improved, lochert05:geographic}; often relies on knowledge of a city street map or predictive traffic pattern; usually targeted at routing around radio obstructions (e.g., buildings)

Geographic routing has also been explored in DTNs in a variety of ways~\cite{soares14:geospray}.

DTN-targeted approaches that complement traditional greedy location routing approaches with a store-and-carry behavior based on predictions from navigation systems~\cite{cheng10:geodtn}; other DTN approaches relax requirements associated with location knowledge to enable geographic based routing when only partial location information is available~\cite{kuiper11:geographical} or rely on infrastucture support (e.g., bus stops) to reliably route based on position information~\cite{park12:position}.