% -----------------------------------------------
%   The Breadcrumb Router
% -----------------------------------------------

The {\sc breadcrumb} router performs geo-source routing using the {\em GeoRouting} extension block, the location coordinates of the node, and the location coordinates of neighboring peers. When it received a bundle with a {\em GeoRouting} block, the router examines the entry at the top of the list, and, if the entry contains a geo entry, the router compares the entry's location against the location coordinates of itself and neighboring peers and decides whether it or a peer is within the specified margin fo the required coordinate. If the router finds a match, it also pops the top entry from the list stored on the {\em GeoRouting} block. Using this strategy, the bundle eventually visits every required location specified in the {\em GeoRouting} block. The name of this router comes from the fact that following these entries back to a particular location is analogous to following breadcrumbs.

The functionality of our implementation of the {\sc breadcrumb} router can be divided into three tasks that fit in the IBR-DTN routing task-queue structure:
\begin{description*}
\item[SearchNextBundle.] When the router receives an event indicating a change in peer connectivity (e.g., a successful peer handshake or a completed bundle transfer), the router computes the next set of bundles to send to the peer.
\item[NextExchange.] This periodic task triggers an exchange of location information with all neighboring peers.
\item[UpdateMyLocation.] This task is queued periodically {\em and} anytime a bundle is received; it updates the router's knowledge of the node's location.
\end{description*}

%{\bf Bundle Filters and Meta Bundles.}
The {\bf SearchNextBundle} and {\bf UpdateMyLocation} tasks both require inspecting the location information packaged in each bundle's {\em GeoRouting} block, which is challenging to do efficiently within IBR-DTN due to the fact that the bundles are kept in persistent storage. On the other hand, maintaining a data structure in memory that contains all the geo information pertaining to each bundle goes against the design of IBR-DTN. We devised an efficient solution that does not require the creation of an additional data-structure and minimizes the amount of bundles that need to pulled from persistent storage.  Our solution relies on two IBR-DTN constructs: {\sc bundle filters} and {\sc meta bundles}. {\sc Meta bundles} are light-weight representations of bundles that contain fields of particular interest. {\sc Bundle filters} query the storage for meta bundles that meet a set of criteria. We added three fields to those already present in a {\sc meta bundle}:
\begin{description*}
\item[hasgeoroute.] A boolean identifier indicating that the bundle has a {\sc GeoRouting} block.
\item[nextgeohop.] The last entry in the {\sc GeoRouting} block.
\item[reacheddest.] A boolean flag indicating that there are no more entries in the {\sc GeoRouting} block (i.e., the bundle has reached its ``final'' destination)
\end{description*}

We also created two {\sc bundle filters}, one pertaining to each task that needs to inspect the bundles:
\begin{description*}
\item[SearchNext.] This filter determines which bundles to send to each peer; it is invoked during the {\sc SearchNextBundle} task. For each {\sc meta bundle} for which {\sc hasgeoroute} is {\sc true}, the filter compares the location of the peer with the {\sc nextgeohop} to see if the peer is closer to it than the host. If so, the meta bundle is added to the list.
\item[UpdateLocation.] This filter determines which bundles need to have their {\sc GeoRouting} blocks updated, i.e., because the location in the top entry has been visited. It bases this decision on whether the location of the node is within the specified margin of error of the {\sc nextgeohop} from the meta bundle.
\end{description*}
Note that forwarding a bundle does not necessarily map one to one with popping a geo entry off of the {\em GeoRouting} block; the {\sc breadcrumb} router implements a relatively standard greedy forwarding behavior in which it optimistically forwards bundles to nodes that are {\em closer} to the next geo waypoint, even if they are not within the specified margin of error of the waypoint.

By using the {\sc bundle filters}, only one lookup into the persistent storage is required; it retrieves a list of {\sc meta bundles} that need action. For the {\sc SearchNextBundle} task, the {\sc meta bundles} contain the information necessary to determine which bundles to transfer. For the {\sc UpdateMyLocation} task, each {\sc meta bundle} represents a bundle that needs to be pulled from persistent storage to have its {\em GeoRouting} block updated. However, the filter limits this retrieval to exactly the bundles that need to have their list of geo-routing entries updated (i.e., by having a satisfied entry popped from the list). This is a considerable improvement over having to retrieve each bundle from storage just to inspect a {\sc GeoRouting} block that in the vast majority of cases will not require updating.

Although the primary functionality of the {\sc breadcrumb} router is geo-source routing, it also doubles as an epidemic router for all bundles that do not contain {\sc GeoRouting} blocks. This functionality was useful so that we could use a single router to both build our trajectories using the {\em GeoTracking} blocks (i.e., to drop the breadcrumbs) and to use {\em GeoRouting} blocks to return to the source (i.e., to follow the breadcrumbs). The router also supports single-copy routing, so that only a single copy of the bundle with a {\em GeoRouting} block persists in the network.

For our initial prototype, the {\sc breadcrumb} router focused on ordered, geo-source routing. Therefore, our router assumes that all {\em GeoRouting} blocks contain entries that must be visited in order, and that each entry pertains to a particular geo location. However, as described, our {\em GeoTracking} and {\em GeoRouting} extension blocks apply more widely; supporting a heterogeneous series of entries would make the router more generalizable since it could be applied to scenarios where the order that locations are visited does not matter, or where a few specific nodes must be visited.

%Although not specifically associated with our router, current IBR-DTN does not support updates to bundles in storage. Therefore, each update to a {\sc GeoRouting} block requires a separate remove and store operation. This is inefficient because it requires that the entire bundle be re-written to storage during every update (this actually makes our modifications to the meta bundles even more critical). Adding this update functionality would improve the performance of our router and of IBR-DTN storage in general.

