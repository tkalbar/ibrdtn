% -----------------------------------------------
%   The Breadcrumb Router
% -----------------------------------------------

The {\sc breadcrumb} router performs geo-source routing using the {\em GeoRouting} extension block, the location of the node, and the location of neighboring peers. When it receives a bundle with a {\em GeoRouting} block, the router examines the entry at the top of the list, and, if the entry contains a geo entry, the router compares the entry's location against its location and the locations of neighboring peers and decides whether it or a peer is within the specified margin of the required coordinate. If the router finds a match, it pops the top entry from the list stored on the {\em GeoRouting} block. The router also forwards bundles without popping entries in the {\em GeoRouting} block if it encounters a peer that is closer to the next required waypoint. %The name of this router comes from the fact that following these entries back to a particular location is analogous to following breadcrumbs.

The {\sc breadcrumb} router's functions are divided into three tasks in the IBR-DTN routing task-queue structure:
\begin{description*}
\item[SearchNextBundle.] Computes the next bundle to send to a peer; invoked when the router receives an event indicating a change in peer connectivity (e.g., a successful peer handshake or completed bundle transfer).
\item[UpdateMyLocation.] Updates the router's knowledge of the node's location; queued periodically and anytime a bundle is received.
\end{description*}

%{\bf Bundle Filters and Meta Bundles.}
{\bf SearchNextBundle} and {\bf UpdateMyLocation} both require inspecting the information in each bundle's {\em GeoRouting} block, which is challenging to do efficiently in IBR-DTN since bundles are kept in persistent storage. On the other hand, maintaining a data structure in memory that contains all the geo information for each bundle is counter to the design of IBR-DTN. We devised a solution that does not require creating an additional data-structure and minimizes the retrieval of bundles from persistent storage.  We rely on two IBR-DTN constructs: {\sc bundle filters} and {\sc meta bundles}. {\sc Meta bundles} are light-weight bundle representations that contain fields of particular interest. {\sc Bundle filters} query the storage for meta bundles that meet a set of criteria. We added three fields to the {\sc meta bundle}:
\begin{description*}
\item[hasgeoroute.] A boolean identifier indicating that the bundle has a {\sc GeoRouting} block.
\item[nextgeohop.] The last entry in the {\sc GeoRouting} block.
\item[reacheddest.] A boolean flag indicating that there are no more entries in the {\sc GeoRouting} block (i.e., the bundle has reached its ``final'' destination)
\end{description*}

We also created two {\sc bundle filters}, one pertaining to each task that needs to inspect the bundles:
\begin{description*}
\item[SearchNext.] Determines which bundles to send to each peer; invoked during {\sc SearchNextBundle}. For each {\sc meta bundle} for which {\sc hasgeoroute} is {\sc true}, the filter compares the location of each peer with the {\sc nextgeohop} to see if the peer is closer to it than the host. If so, the meta bundle is added to the list.
\item[UpdateLocation.] Determines which {\sc GeoRouting} blocks need updating; the decision is based on whether the location of the node is within the specified margin of error of the {\sc nextgeohop} from the meta bundle.
\end{description*}
Forwarding a bundle does not necessarily pop a geo entry off of the {\em GeoRouting} block; the {\sc breadcrumb} router also greedily forwards bundles to nodes that are {\em closer} to the next geo waypoint, even if they are not within the specified margin of error of the waypoint.

By using the {\sc bundle filters}, only one lookup into the persistent storage is required; it retrieves a list of {\sc meta bundles} that need action. For {\sc SearchNextBundle}, the {\sc meta bundles} contain the information necessary to determine which bundles to transfer. For {\sc UpdateMyLocation}, each {\sc meta bundle} represents a bundle that needs to be pulled from persistent storage to have its {\em GeoRouting} block updated. 
%The filter limits this retrieval to exactly the bundles that need to have their list of geo-routing entries updated (i.e., by having a satisfied entry popped from the list). 
This is a considerable improvement over retrieving every bundle just to inspect a {\em GeoRouting} block that, in the majority of cases, will not require updating.

The {\sc breadcrumb} router defaults to epidemic routing when bundles do not contain {\sc GeoRouting} blocks. We can therefore use a single router to build trajectories using {\em GeoTracking} blocks (i.e., to drop breadcrumbs) and to use {\em GeoRouting} blocks to return to the source (i.e., to follow breadcrumbs). The router supports single-copy routing, so that only a single copy of the bundle with a {\em GeoRouting} block persists in the network. For our prototype, the router provides ordered, geo-source routing, i.e., it assumes that all {\em GeoRouting} blocks contain entries that must be visited in order, and that each entry pertains to a particular geo location. Our {\em GeoTracking} and {\em GeoRouting} extension blocks apply more widely; supporting a heterogeneous series of entries would make the router more generalizable since it could be applied to scenarios where the order that locations are visited does not matter, or where a few specific nodes must be visited.

%Although not specifically associated with our router, current IBR-DTN does not support updates to bundles in storage. Therefore, each update to a {\sc GeoRouting} block requires a separate remove and store operation. This is inefficient because it requires that the entire bundle be re-written to storage during every update (this actually makes our modifications to the meta bundles even more critical). Adding this update functionality would improve the performance of our router and of IBR-DTN storage in general.

