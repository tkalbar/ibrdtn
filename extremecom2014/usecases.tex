% -----------------------------------------------
%
%   Use Cases
%
% -----------------------------------------------

{\bf Use Cases.} There are myriad uses for our bundle extensions and
router. We briefly describe three examples.
% that motivate tracking,
%geo-routing, and their combination.

{\it Safe Data Sharing.} Tracking bundles is clearly widely useful for
debugging and general spatiotemporal provenance. As a concrete
example, a secure application may want to track the movement of
bundles through space to ensure that bundles (and their content) never
leave a pre-specified ``safe'' zone~\cite{michel12:spatiotemporal}. By
placing a {\em GeoTracking} extension on each bundle, the application
can compare the trajectory of the bundle against the safe zone to
ensure the desired property.

{\it An Oil Spill.} Consider an environmental disaster, e.g., an oil
spill, detected by one or more distributed sensors. Based on the
location of the spill and simultaneously sensed ambient information
(e.g., water currents, winds, etc.), the sensor can compute where the
spill is likely to dissipate and generate bundles that can be
explicitly routed along the trajectories of dissipation. As the
bundles propagate, their routing paths (i.e., trajectories) can be
updated by devices they pass through based on locally sensed ambient
conditions (e.g., changes in water currents or winds). Our {\em
  GeoRouting} extension to the Bundle protocol's block format,
combined with our {\sc breadcrumb} router, can achieve this behavior.

{\it A Maze.} 
%As described previously, we provide a concrete way to bring tracking and routing together; this scenario is a specific instantiation of a much more general notion of location-based publish-subscribe in DTNs. 
Consider a prisoner who is held in a maze, which is patrolled by guards whose devices can act as intermediate DTN nodes. The prisoner sends a bundle with a destination of a known rescuer outside of the maze. This bundle tracks its trajectory, logging a successful path out of the maze. The rescuer can send a response back to the prisoner, routed along the geo-waypoints in the maze. The response may contain information about the perils of the exit path, sensed as the prisoner's original message traversed to the rescuer. Using our {\sc breadcrumb} router, the rescuer's return bundle can reach the prisoner regardless of whether the particular guards in the maze change.
%, as long as there exists enough coverage of the maze path by {\em some} DTN nodes.






