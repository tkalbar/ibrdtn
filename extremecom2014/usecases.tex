% -----------------------------------------------
%
%   Use Cases
%
% -----------------------------------------------

{\bf Use Cases.} As indicated in the introduction, there are myriad
uses for the {\em GeoTracking} and {\em GeoRouting} bundle extensions,
as well as for the {\sc breadcrumb} router itself. In this section, we
briefly take three examples that separate the motivations for bundle
tracking, bundle geo-routing, and the combination of the two.

{\it Safe Data Sharing.} Tracking bundles as they move through a DTN
is clearly widely useful for debugging and for general spatiotemporal
provenance. As another concrete example, a secure application may want
to control (and then track) the movement of bundles through space to
ensure that the bundle (and its content) never leave a pre-specified
``safe'' zone~\cite{michel12:spatiotemporal}. By placing a {\em
  GeoTracking} extension on each bundle, the application can compare
the trajectory of the bundle against the safe zone to ensure the
desired property.

{\it An Oil Spill.} Consider the case of some natural disaster, e.g.,
an oil spill, detected by one or more distributed sensors. Based on
the location of the spill and simultaneously sensed ambient
information (e.g., water currents, winds, etc.), the sensor can
compute where the spill is likely to dissipate and generate
bundles that can be explicitly routed along the trajectory or
trajectories of dissipation. As the bundles propagate, their routing
paths (i.e., trajectories) can be updated by devices they pass through
based on locally sensed ambient conditions (e.g., changes in water
currents or winds). Our {\em GeoRouting} extension to the Bundle
protocol's block format, as well as our {\sc breadcrumb} router, which
interprets this extension, can achieve this routing behavior.

{\it A Maze.} To bring tracking and routing together in one package,
consider a scenario in which a prisoner is held in a maze, and the
maze is patrolled by guards who can act as intermediate DTN nodes. The
prisoner sends a bundle with a destination of a known rescuer outside
of the maze. This bundle tracks its trajectory, effectively logging a
successful path out of the maze. The rescuer can then send a response
back to the prisoner, routed along the waypoints of the turns in the
maze. The response may contain information about the perils of the
exit path, sensed as the prisoners original message traversed to the
rescuer. Using our {\sc breadcrumb} router, the rescuer's return
bundle can reach the prisoner regardless of whether the particular
guards in the maze change, as long as there exist enough coverage of
the maze path by {\em some} DTN nodes.






